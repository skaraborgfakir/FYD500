\documentclass[a4paper]{article}

\begin{document}

\begin{verbatim}
top/ps enbart de processer som delar
   tty med det skal som ps körs via

8379 (skalets PID)
pts/2
ps -e / ps aux

find / --- det finns mycket data i Chalmers datornät
       --- i varje fall listas alla filer som min användare kommer åt
       --- i nätverket... sys:procenten i top gick upp till 40-50
       --- datorn jobbar så hårt som den kan beroende på hur snabbt
       --- nätverket är...
\caption{ps, tty och find}
\end{verbatim}


\begin{figure}
\centering
\begin{verbatim}
[stenis@16 ~]$ xeyes &
[1] 24273 
[stenis@16 ~]$ xclock &
[2] 2427
[stenis@16 ~]$ kill %1 %2
[stenis@16 ~]$ 
\end{verbatim}
\caption{kommandot 'xeyes och xclock med ''kill''}
\end{figure}


\begin{figure}
\centering
\begin{verbatim}
  PID TTY          TIME CMD
    1 ?        00:00:11 systemd
    2 ?        00:00:00 kthreadd
    3 ?        00:00:00 ksoftirqd/0
    5 ?        00:00:00 kworker/0:0H
    7 ?        00:00:01 migration/0
    8 ?        00:00:00 rcu_bh
    9 ?        00:01:46 rcu_sched
   10 ?        00:00:00 lru-add-drain
   11 ?        00:00:00 watchdog/0
   12 ?        00:00:00 watchdog/1
   13 ?        00:00:01 migration/1
   14 ?        00:00:00 ksoftirqd/1
   16 ?        00:00:00 kworker/1:0H
   17 ?        00:00:00 watchdog/2
   18 ?        00:00:01 migration/2
   19 ?        00:00:00 ksoftirqd/2
   21 ?        00:00:00 kworker/2:0H
   22 ?        00:00:00 watchdog/3
   23 ?        00:00:01 migration/3
   24 ?        00:00:05 ksoftirqd/3
   26 ?        00:00:00 kworker/3:0H
   28 ?        00:00:00 kdevtmpfs
   29 ?        00:00:00 netns
   30 ?        00:00:00 khungtaskd
   31 ?        00:00:00 writeback
   32 ?        00:00:00 kintegrityd
   33 ?        00:00:00 bioset
   34 ?        00:00:00 bioset
   35 ?        00:00:00 bioset
   36 ?        00:00:00 kblockd
   37 ?        00:00:00 md
   38 ?        00:00:00 edac-poller
   39 ?        00:00:00 watchdogd
   45 ?        00:00:00 kswapd0
   46 ?        00:00:00 ksmd
   47 ?        00:00:00 khugepaged
   48 ?        00:00:00 crypto
   56 ?        00:00:00 kthrotld
   59 ?        00:00:00 kmpath_rdacd
   60 ?        00:00:00 kaluad
   62 ?        00:00:00 kpsmoused
   64 ?        00:00:00 ipv6_addrconf
   77 ?        00:00:00 deferwq
  111 ?        00:00:00 kauditd
  267 ?        00:00:00 rpciod
  268 ?        00:00:00 xprtiod
  533 ?        00:00:26 firefox
  602 ?        00:00:00 iscsi_eh
  778 pts/1    00:00:00 xdvi-xaw
  942 ?        00:00:00 ata_sff
 1054 ?        00:00:00 scsi_eh_0
 1059 ?        00:00:00 scsi_tmf_0
 1064 ?        00:00:00 scsi_eh_1
 1069 ?        00:00:00 scsi_tmf_1
 1080 ?        00:00:00 scsi_eh_2
 1092 ?        00:00:00 scsi_tmf_2
 1112 ?        00:00:00 scsi_eh_3
 1115 ?        00:00:00 scsi_tmf_3
 1147 ?        00:00:09 Web Content
 1710 ?        00:00:00 afs_pagecopy
 1721 ?        00:00:00 afs_rxlistener
 1723 ?        00:00:00 afsd
 1724 ?        00:00:02 afs_rxevent
 1726 ?        00:00:00 afs_callback
 1730 ?        00:00:00 afsd
 1731 ?        00:00:00 afs_checkserver
 1732 ?        00:00:00 afs_background
 1736 ?        00:00:00 afs_background
 1737 ?        00:00:00 afs_background
 1738 ?        00:00:00 afs_background
 1740 ?        00:00:00 afs_cachetrim
 1747 ?        00:00:00 sshd
 1865 ?        00:00:08 sh
 1868 ?        00:00:00 gdm
 1878 ?        00:00:00 accounts-daemon
 1963 ?        00:00:00 upowerd
 2343 ?        00:00:00 boltd
 2345 ?        00:00:00 packagekitd
 2353 ?        00:00:00 wpa_supplicant
 2415 ?        00:00:00 colord
 2426 ?        00:00:00 sleep
 2442 ?        00:00:31 pcscd
 2609 ?        00:00:01 kworker/0:1H
 2610 ?        00:00:01 kworker/1:1H
 2638 ?        00:00:00 kworker/u9:0
 2639 ?        00:00:00 i915/signal:0
 2640 ?        00:00:00 i915/signal:1
 2641 ?        00:00:00 i915/signal:2
 2642 ?        00:00:00 i915/signal:6
 2732 ?        00:00:00 jbd2/sda5-8
 2733 ?        00:00:00 ext4-rsv-conver
 2835 ?        00:00:00 systemd-journal
 2869 ?        00:00:00 lvmetad
 2884 ?        00:00:00 systemd-udevd
 3323 ?        00:00:00 sleep
 3325 ?        00:00:00 ps
 3725 ?        00:00:07 emacs
 4083 ?        00:00:00 irq/128-mei_me
 4444 ?        00:00:00 nvidia-modeset
 4532 ?        00:00:00 kvm-irqfd-clean
 4698 ?        00:00:00 jbd2/sda3-8
 4715 ?        00:00:00 ext4-rsv-conver
 5446 ?        00:00:00 jbd2/sda6-8
 5457 ?        00:00:00 ext4-rsv-conver
 5684 ?        00:00:00 auditd
 5686 ?        00:00:00 audispd
 5688 ?        00:00:00 sedispatch
 5720 ?        00:00:06 udisksd
 5724 ?        00:00:05 irqbalance
 5727 ?        00:00:00 lsmd
 5728 ?        00:00:01 gpm
 5732 ?        00:00:02 polkitd
 5733 ?        00:00:06 rngd
 5735 ?        00:00:01 dbus-daemon
 5737 ?        00:00:05 chronyd
 5777 ?        00:00:00 acpid
 5785 ?        00:00:00 rpcbind
 5836 ?        00:00:00 smartd
 5839 ?        00:00:00 rtkit-daemon
 5843 ?        00:00:00 alsactl
 5844 ?        00:00:00 systemd-logind
 5846 ?        00:00:00 ModemManager
 5865 ?        00:00:45 nscd
 5866 ?        00:00:00 mcelog
 5881 ?        00:00:00 NetworkManager
 5900 ?        00:00:02 kworker/3:1H
 5901 ?        00:00:02 kworker/2:1H
 5903 ?        00:00:00 ksmtuned
 5905 ?        00:00:00 gssproxy
 5933 ?        00:00:00 rpc.gssd
 5960 ?        00:00:00 denyhosts.py
 6049 ?        00:00:00 dhclient
 6294 ?        00:00:06 tuned
 6296 ?        00:00:02 rsyslogd
 6302 ?        00:00:00 libvirtd
 6304 ?        00:00:00 rhsmcertd
 6306 ?        00:00:00 atd
 6307 ?        00:00:00 crond
 6340 ?        00:00:02 automount
 6405 ?        00:01:14 venus
 6624 ?        00:00:00 unbound
 6625 ?        00:00:00 dnssec-triggerd
 6641 ?        00:00:00 rpc.statd
 6664 ?        00:00:01 kworker/1:3
 6779 ?        00:00:00 dnsmasq
 6780 ?        00:00:00 dnsmasq
 6823 ?        00:00:00 master
 6825 ?        00:00:00 qmgr
 7443 ?        00:00:00 libc.so
 7542 pts/1    00:00:00 bash
 8109 ?        00:00:00 kworker/2:1
 8377 pts/0    00:00:00 screen
 8378 ?        00:00:00 screen
 8379 pts/2    00:00:00 bash
 8627 ?        00:00:00 nfsiod
 8654 ?        00:00:00 nfsv4.1-svc
 8790 pts/3    00:00:00 bash
 8881 pts/3    00:00:01 top
11034 ?        00:00:03 Web Content
11064 pts/2    00:00:00 vim
14551 tty2     00:00:00 agetty
14555 tty3     00:00:00 agetty
14557 tty4     00:00:00 agetty
16227 tty1     00:00:24 X
16229 ?        00:01:49 irq/130-nvidia
16230 ?        00:00:00 nvidia
16809 ?        00:00:00 kworker/0:1
17147 ?        00:00:00 kworker/1:1
17567 ?        00:00:00 kworker/2:0
20578 ?        00:00:00 Web Content
21336 ?        00:00:00 kworker/3:1
21346 ?        00:00:00 gvfsd-network
21469 ?        00:00:00 gvfsd-dnssd
23170 ?        00:00:00 gdm-session-wor
23432 ?        00:00:00 pickup
24739 ?        00:00:00 kworker/3:0
24760 ?        00:00:00 gnome-keyring-d
24766 ?        00:00:00 ibus-daemon
24769 ?        00:00:00 ibus-dconf
24771 ?        00:00:00 ibus-x11
24783 ?        00:00:00 gnome-session-b
24792 ?        00:00:00 dbus-launch
24793 ?        00:00:00 dbus-daemon
24861 ?        00:00:00 imsettings-daem
24866 ?        00:00:00 gvfsd
24871 ?        00:00:00 gvfsd-fuse
25055 ?        00:00:00 ssh-agent
25147 ?        00:00:00 at-spi-bus-laun
25152 ?        00:00:00 dbus-daemon
25154 ?        00:00:00 at-spi2-registr
25183 ?        00:01:10 gnome-shell
25509 ?        00:00:00 pulseaudio
25520 ?        00:00:00 kworker/u8:1
25529 ?        00:00:00 ibus-daemon
25538 ?        00:00:00 xdg-permission-
25543 ?        00:00:00 gnome-shell-cal
25549 ?        00:00:00 evolution-sourc
25564 ?        00:00:00 dconf-service
25566 ?        00:00:00 ibus-dconf
25571 ?        00:00:00 ibus-x11
25575 ?        00:00:00 ibus-portal
25592 ?        00:00:00 mission-control
25595 ?        00:00:00 gvfs-udisks2-vo
25600 ?        00:00:00 goa-daemon
25608 ?        00:00:00 gvfs-afc-volume
25619 ?        00:00:00 gvfs-gphoto2-vo
25623 ?        00:00:00 goa-identity-se
25629 ?        00:00:00 gvfs-mtp-volume
25637 ?        00:00:00 gvfs-goa-volume
25681 ?        00:00:00 gsd-account
25685 ?        00:00:00 gsd-power
25686 ?        00:00:00 gsd-print-notif
25688 ?        00:00:00 gsd-rfkill
25689 ?        00:00:00 gsd-screensaver
25690 ?        00:00:00 gsd-sharing
25692 ?        00:00:00 gsd-smartcard
25700 ?        00:00:00 gsd-sound
25702 ?        00:00:00 gsd-xsettings
25704 ?        00:00:00 gsd-wacom
25714 ?        00:00:00 gsd-a11y-settin
25717 ?        00:00:00 gsd-clipboard
25722 ?        00:00:00 gsd-color
25723 ?        00:00:00 gsd-datetime
25724 ?        00:00:00 gsd-housekeepin
25725 ?        00:00:00 gsd-keyboard
25727 ?        00:00:00 gsd-media-keys
25734 ?        00:00:00 gsd-mouse
25751 ?        00:00:00 gsd-printer
25801 ?        00:00:00 nautilus-deskto
25806 ?        00:00:00 gvfsd-trash
25820 ?        00:00:00 dnssec-trigger
25822 ?        00:00:00 tracker-extract
25826 ?        00:00:00 tracker-miner-a
25832 ?        00:00:00 evolution-alarm
25840 ?        00:00:00 tracker-miner-f
25841 ?        00:00:00 tracker-store
25848 ?        00:00:00 clevis-luks-udi
25849 ?        00:00:00 tracker-miner-u
25853 ?        00:00:00 gsd-disk-utilit
25901 ?        00:00:00 clevis-luks-udi
25949 ?        00:00:00 evolution-calen
25950 ?        00:00:00 escd
25994 ?        00:00:00 ibus-engine-sim
26015 ?        00:00:00 evolution-calen
26047 ?        00:00:00 evolution-addre
26060 ?        00:00:00 gvfsd-metadata
26089 ?        00:00:00 evolution-addre
26134 ?        00:00:00 gvfsd-burn
27269 ?        00:00:00 kworker/0:2
27282 ?        00:00:00 gconfd-2
27466 ?        00:00:00 kworker/u8:5
28013 ?        00:00:01 xterm
28015 pts/0    00:00:00 bash
30017 ?        00:00:00 kworker/0:0
32531 ?        00:00:00 kworker/u8:0
32700 ?        00:00:00 kworker/1:0
\end{verbatim}
\caption{kommandot 'ps -e' i terminal}
\end{figure}


\begin{figure}
\begin{verbatim}
[stenis@16 ~]$ 
Message from stenis@16.f7203.studat.rh75.ii.htb on pts/3 at 19:40 ...
[    0.000000] Initializing cgroup subsys cpuset
[    0.000000] Initializing cgroup subsys cpu
[    0.000000] Initializing cgroup subsys cpuacct
[    0.000000] Linux version 3.10.0-957.10.1.el7.x86_64 (mockbuild@x86-040.build.eng.bos
[    0.000000] Command line: BOOT_IMAGE=/vmlinuz-3.10.0-957.10.1.el7.x86_64 root=/dev/sd
[    0.000000] e820: BIOS-provided physical RAM map:
[    0.000000] BIOS-e820: [mem 0x0000000000000000-0x000000000009c3ff] usable
[    0.000000] BIOS-e820: [mem 0x000000000009c400-0x000000000009ffff] reserved
[    0.000000] BIOS-e820: [mem 0x00000000000e0000-0x00000000000fffff] reserved
[    0.000000] BIOS-e820: [mem 0x0000000000100000-0x00000000859a3fff] usable
[    0.000000] BIOS-e820: [mem 0x00000000859a4000-0x00000000859a4fff] ACPI NVS
[    0.000000] BIOS-e820: [mem 0x00000000859a5000-0x00000000859a5fff] reserved
[    0.000000] BIOS-e820: [mem 0x00000000859a6000-0x0000000093466fff] usable
[    0.000000] BIOS-e820: [mem 0x0000000093467000-0x00000000936a5fff] reserved
[    0.000000] BIOS-e820: [mem 0x00000000936a6000-0x00000000936e8fff] ACPI data
[    0.000000] BIOS-e820: [mem 0x00000000936e9000-0x0000000094009fff] ACPI NVS
[    0.000000] BIOS-e820: [mem 0x000000009400a000-0x00000000945fefff] reserved
[    0.000000] BIOS-e820: [mem 0x00000000945ff000-0x00000000945fffff] usable
[    0.000000] BIOS-e820: [mem 0x0000000094600000-0x0000000097ffffff] reserved
[    0.000000] BIOS-e820: [mem 0x00000000f8000000-0x00000000fbffffff] reserved
[    0.000000] BIOS-e820: [mem 0x00000000fe000000-0x00000000fe010fff] reserved
[    0.000000] BIOS-e820: [mem 0x00000000fec00000-0x00000000fec00fff] reserved
[    0.000000] BIOS-e820: [mem 0x00000000fee00000-0x00000000fee00fff] reserved
[    0.000000] BIOS-e820: [mem 0x00000000ff000000-0x00000000ffffffff] reserved
[    0.000000] BIOS-e820: [mem 0x0000000100000000-0x0000000865ffffff] usable
[    0.000000] NX (Execute Disable) protection: active
[    0.000000] SMBIOS 3.0 present.
[    0.000000] DMI: Dell Inc. Precision Tower 3620/0MWYPT, BIOS 2.7.3 01/31/2018
[    0.000000] e820: update [mem 0x00000000-0x00000fff] usable ==> reserved
[    0.000000] e820: remove [mem 0x000a0000-0x000fffff] usable
[    0.000000] e820: last_pfn = 0x866000 max_arch_pfn = 0x400000000
[    0.000000] MTRR default type: write-back
[    0.000000] MTRR fixed ranges enabled:
[    0.000000]   00000-9FFFF write-back
[    0.000000]   A0000-BFFFF uncachable
[    0.000000]   C0000-FFFFF write-protect
[    0.000000] MTRR variable ranges enabled:
[    0.000000]   0 base 00C0000000 mask 7FC0000000 uncachable
[    0.000000]   1 base 00A0000000 mask 7FE0000000 uncachable
...
..
[   84.007067] nvidia 0000:01:00.0: irq 130 for MSI/MSI-X
[   84.854870] nvidia-modeset: Allocated GPU:0 (GPU-b0717a24-999a-e2f4-cf
[ 1046.116274] fuse init (API version 7.22)
[ 2802.768066] traps: TaskSchedulerSi[29816] trap int3 ip:563149e0eebe sp
[ 2803.323018] nvidia-modeset: Freed GPU:0 (GPU-b0717a24-999a-e2f4-cf94-2
[ 2803.391135] nvidia 0000:01:00.0: irq 130 for MSI/MSI-X
[ 2804.054975] nvidia-modeset: Allocated GPU:0 (GPU-b0717a24-999a-e2f4-cf
\end{verbatim}
\caption{kommandot 'dmesg | write stenis pts/2' från pts/3}
\end{figure}
Vilket är vad kärnan hittar om vad som finns i form av hårdvara i maskinen.

\begin{figure}
\begin{verbatim}
[stenis@16 ~]$ uptime
19:46:48 up 19:12,  5 users,  load average: 0.01, 0.05, 0.12
[stenis@16 ~]$ tty
/dev/pts/4
\end{verbatim}
\caption{uptime och tty}
\end{figure}


\begin{figure}
\begin{verbatim}
[stenis@16 ~]$ ls -l /usr/bin/|grep '^-..s' 
-rwsr-xr-x.   1 root root       52952 Sep 14  2017 at
-rwsr-xr-x.   1 root root       64232 Jun 28  2016 chage
-rws--x--x.   1 root root       24048 Feb  2  2018 chfn
-rws--x--x.   1 root root       23960 Feb  2  2018 chsh
\end{verbatim}
\caption{SUID kommandon}
\end{figure}

chfn kan ändra de ``lite längre personuppgifterna'' om en användare i Linux
och behöver därför kunna ändra i /etc/passwd. Den fil som förr räknade upp en maskins
användare.

chsh ändrar vilket skal som en användare använder, på samma vis som chfn.
/etc/passwd är skrivskyddad för alla förutom systemadministratör.

Vem som är största tidstjuven --- gnome-shell eller firefox (webbläsaren.)

Ja, jag kan köra reboot --- jag får göra så troligtvis därför att maskinen enbart används
av en person åt gången.

\end{document}
